% !TeX document-id = {31322199-07d0-4e7b-ac39-8f4bff032313}
% !TeX root = workshop-codemash-2023.tex
% !TeX TXS-program:compile = txs:///pdflatex/[--shell-escape]
% !TeX encoding = UTF-8
% !TeX spellcheck = en_US
% https://orcid.org/0000-0003-4586-8500
% session details: https://www.codemash.org/session-details/?id=375030

% Other possible values are: 1610, 149, 54, 43 and 32.
% By default, it is to 128mm by 96mm(4:3)
\documentclass[aspectratio=169]{beamer}

%\setbeamertemplate{headline navigation symbols}{}         % no navigation symbols
\usetheme{Warsaw}
\usecolortheme{seahorse}
\usepackage[absolute,overlay]{textpos} % Text positioning

%%%%%%%%%%%%%%%%%%% TEXT COLOR %%%%%%%%%%%%%%%%%
\usepackage{xcolor}
\definecolor{olive}{rgb}{0.3, 0.4, .1}
\definecolor{fore}{RGB}{249,242,215}
\definecolor{back}{RGB}{51,51,51}
\definecolor{title}{RGB}{255,0,90}
\definecolor{dgreen}{rgb}{0.,0.6,0.}
\definecolor{gold}{rgb}{1.,0.84,0.}
\definecolor{JungleGreen}{cmyk}{0.99,0,0.52,0}
\definecolor{BlueGreen}{cmyk}{0.85,0,0.33,0}
\definecolor{RawSienna}{cmyk}{0,0.72,1,0.45}
\definecolor{Magenta}{cmyk}{0,1,0,0}

%Information to be included in the title page:
\title{Build a Serverless Github Bot in GCP}
\subtitle{}
\author{Franklin Diaz}
\institute{DE:AD:10:C5}
\date{Tuesday January 10, 2023}

%%%%%%%%%%%%%%%%%%% SLIDE VIEW SETTINGS %%%%%%%%%%%%%%%%%
\hypersetup{pdfpagemode=FullScreen} % put the deck in presentation mode
% this is to make notes mode work with pympress package
\usepackage{pgfpages}
\setbeameroption{show notes on second screen}



\begin{document}

%%% Title Slide %%%
\frame{\titlepage}


%\begin{frame}
%        \frametitle{Table of Contents}
%        \tableofcontents
%\end{frame}

% uncomment one of these for the whole doc, or add at the start of a section as desired                                                                 
%\usebackgroundtemplate{\includegraphics[width=\paperwidth]{../images/field.jpg}}
\usebackgroundtemplate{\includegraphics[width=\paperwidth]{../images/landscape.jpg}}
%\usebackgroundtemplate{\includegraphics[width=\paperwidth]{../images/tree.jpg}}


\section{INTRODUCTION}
\begin{frame}
	\Huge \textcolor{dgreen}{INTRODUCTION}
\end{frame}

% you can uncomment one of these for the whole doc, or add at the start of each section as desired                                                                            
\usebackgroundtemplate{\includegraphics[width=\paperwidth]{../images/field.jpg}}
%\usebackgroundtemplate{\includegraphics[width=\paperwidth]{../images/landscape.jpg}}
%\usebackgroundtemplate{\includegraphics[width=\paperwidth]{../images/tree.jpg}}

\begin{frame}
	\frametitle{Overview: Usage}
	The big picture for operation.
	\vspace{2mm}

	\includegraphics[width=0.785\textwidth]{../images/arch_diagrams-big-block.png}

\end{frame}

\begin{frame}
	\frametitle{Overview: Deployment}
	The big picture for deployment.
	\vspace{2mm}
\end{frame}

\begin{frame}
	\frametitle{Outline}
	A high level overview of the learning path is as follows:
	\begin{raggedright}
		\begin{itemize}
			\item Prerequisites
			\item Github setup.
			\item Set up a development environment.
			\item Review the Python source for the bot.
			\item Configure Terraform and deploy the bot.
			\item Test it out.
			\item Explore possibilities for extending the functionality.
		\end{itemize}
	\end{raggedright}
	\vspace{2mm}
\end{frame}

% uncomment one of these for the whole doc, or add at the start of a section as desired                                                                 
%\usebackgroundtemplate{\includegraphics[width=\paperwidth]{../images/field.jpg}}
\usebackgroundtemplate{\includegraphics[width=\paperwidth]{../images/landscape.jpg}}
%\usebackgroundtemplate{\includegraphics[width=\paperwidth]{../images/tree.jpg}}


\section{PRE-WORK}
\begin{frame}
	\Huge \textcolor{dgreen}{PRE-WORK}
\end{frame}

% you can uncomment one of these for the whole doc, or add at the start of each section as desired                                                                            
\usebackgroundtemplate{\includegraphics[width=\paperwidth]{../images/field.jpg}}
%\usebackgroundtemplate{\includegraphics[width=\paperwidth]{../images/landscape.jpg}}
%\usebackgroundtemplate{\includegraphics[width=\paperwidth]{../images/tree.jpg}}

\begin{frame}
	\frametitle{Setup: VSCode}
	VSCode (\href{https://code.visualstudio.com}{https://code.visualstudio.com})
	\begin{itemize}
		\item Windows 64 bit User Installer: \href{https://prereqs.codemash.org/Files/VVSCodeUserSetup-x64-1.73.1.exe}{VSCodeUserSetup-x64-1.73.1.exe}
		\item Mac Universal: \href{https://prereqs.codemash.org/Files/VSCode-darwin-universal.zip}{VSCode-darwin-universal.zip}
		\item Linux (Debian, Ubuntu): \href{https://prereqs.codemash.org/Files/code_1.73.1-1667967334_amd64.deb}{code\_1.73.1-1667967334\_amd64.deb}
		\item  Linux (Red Hat, Fedora, SUSE): \href{https://prereqs.codemash.org/Files/code-1.73.1-1667967421.el7.x86_64.rpm}{code-1.73.1-1667967421.el7.x86\_64.rpm}
	\end{itemize}
	\vspace{2mm}
	\href{https://code.visualstudio.com/docs/devcontainers/containers}{Click this link for details on using dev containers in VSCode}
\end{frame}

\begin{frame}
	\frametitle{Setup: git}
	GIT (\href{https://git-scm.com/downloads}{https://git-scm.com/downloads})
	\begin{itemize}
		\item Windows 32 Bit: \href{https://prereqs.codemash.org/Files/Git-2.38.1-64-bit.exe}{Git-2.38.1-64-bit.exe}
		\item Windows 64 Bit: \href{https://prereqs.codemash.org/Files/Git-2.38.1-32-bit.exe}{Git-2.38.1-32-bit.exe}
		\item Mac: \href{https://prereqs.codemash.org/Files/git-2.15.0-intel-universal-mavericks.dmg}{git-2.15.0-intel-universal-mavericks.dmg}
	\end{itemize}
\end{frame}

\begin{frame}
	\frametitle{Setup: Docker Desktop}
	Docker Desktop (\href{https://www.docker.com/}{https://www.docker.com/})

	\begin{itemize}
		\item Windows: \href{https://prereqs.codemash.org/Files/Docker\%20Desktop\%20Installer.exe}{Docker Desktop Installer.exe}
		\item MacOS (Intel Chip): \href{https://prereqs.codemash.org/Files/Docker.dmg}{Docker.dmg}
		\item MacOS (M1 Chip): \href{https://prereqs.codemash.org/Files/Chip/Docker.dmg}{Docker.dmg}
		\item Linux instructions can be found: \href{https://docs.docker.com/desktop/install/linux-install/}{here}
	\end{itemize}
	\vspace{2mm}

	\href{https://code.visualstudio.com/docs/devcontainers/containers\#\_installation}{Click here to see Docker setup steps from Microsoft}

\end{frame}

\note[itemize]{
	\item this is a test
}

\begin{frame}
	\frametitle{Setup: Clone and Open the Project Repository}
	\begin{itemize}
		\item Time to clone the repository.
		\item \href{https://github.com/devsecfranklin/workshop-codemash-2023}{Click this link for the Github repository}
		\item In VSCode, press F1 and enter the command ``Dev Containers: Open Folder in Container''
		\begin{itemize}
			\item You can also choose ``Dev Containers: Open Workspace in Container''
			\item \href{https://code.visualstudio.com/docs/devcontainers/tutorial}{Here is the Microsoft VSCode dev containers tutorial}
		\end{itemize}
		\item From the top menu select ``Terminal -- New Terminal''
		\item Now ``cd /workspaces/workshop-codemash-2023/bin'' and type ``setup-dev-env.sh''
	\end{itemize}
	\vspace{2mm}
\end{frame}

\note[itemize]{
	\item this is a test
}

\begin{frame}
	\frametitle{Google Cloud: Account Setup}
	\begin{itemize}
		\item \href{https://cloud.google.com/free}{Sign up for a free tier GCP account}.
		\item Navigate to \href{https://cloud.google.com/}{https://cloud.google.com/} and make sure you have a usable project to work in.
		\item \href{https://cloud.google.com/resource-manager/docs/creating-managing-projects}{Here is some infomration about creating projects in GCP}
		\item Add your project name in the file ``/workspaces/workshop-codemash-2023/.envrc''
		\item Type the command ``direnv allow .'' to reload the ENV variables.
		\item In the dev container, run the command ``gcloud auth login'' and follow the directions there.
		\item Verify you are connected to GCP with the command ``gcloud auth list''	
\end{itemize}
\end{frame}

\begin{frame}
	\frametitle{Google Cloud: Create Service User}
	
	We create a service user in GCP with limited scope of permissions.
	
\end{frame}

% uncomment one of these for the whole doc, or add at the start of a section as desired                                                                 
%\usebackgroundtemplate{\includegraphics[width=\paperwidth]{../images/field.jpg}}
\usebackgroundtemplate{\includegraphics[width=\paperwidth]{../images/landscape.jpg}}
%\usebackgroundtemplate{\includegraphics[width=\paperwidth]{../images/tree.jpg}}


\section{PYTHON}
\begin{frame}
	\Huge \textcolor{dgreen}{PYTHON}
\end{frame}

% you can uncomment one of these for the whole doc, or add at the start of each section as desired                                                                            
\usebackgroundtemplate{\includegraphics[width=\paperwidth]{../images/field.jpg}}
%\usebackgroundtemplate{\includegraphics[width=\paperwidth]{../images/landscape.jpg}}
%\usebackgroundtemplate{\includegraphics[width=\paperwidth]{../images/tree.jpg}}


\begin{frame}
	\frametitle{The Python Application}

	Discuss the code for the cloud function, see how all that works.

\end{frame}

% uncomment one of these for the whole doc, or add at the start of a section as desired                                                                 
%\usebackgroundtemplate{\includegraphics[width=\paperwidth]{../images/field.jpg}}
\usebackgroundtemplate{\includegraphics[width=\paperwidth]{../images/landscape.jpg}}
%\usebackgroundtemplate{\includegraphics[width=\paperwidth]{../images/tree.jpg}}


\section{TERRAFORM}
\begin{frame}
	\Huge \textcolor{dgreen}{TERRAFORM}
\end{frame}

% you can uncomment one of these for the whole doc, or add at the start of each section as desired                                                                            
\usebackgroundtemplate{\includegraphics[width=\paperwidth]{../images/field.jpg}}
%\usebackgroundtemplate{\includegraphics[width=\paperwidth]{../images/landscape.jpg}}
%\usebackgroundtemplate{\includegraphics[width=\paperwidth]{../images/tree.jpg}}

\begin{frame}
	\frametitle{The Terraform Installer}

	We use Terraform to automate the Cloud Function installation.

\end{frame}

\begin{frame}
	\frametitle{Deploying with Terraform}

	Let's do a Terraform deployment.

\end{frame}

% uncomment one of these for the whole doc, or add at the start of a section as desired                                                                 
%\usebackgroundtemplate{\includegraphics[width=\paperwidth]{../images/field.jpg}}
\usebackgroundtemplate{\includegraphics[width=\paperwidth]{../images/landscape.jpg}}
%\usebackgroundtemplate{\includegraphics[width=\paperwidth]{../images/tree.jpg}}


\section{EXTRA}
\begin{frame}
	\Huge \textcolor{dgreen}{EXTRA}
\end{frame}

% you can uncomment one of these for the whole doc, or add at the start of each section as desired                                                                            
\usebackgroundtemplate{\includegraphics[width=\paperwidth]{../images/field.jpg}}
%\usebackgroundtemplate{\includegraphics[width=\paperwidth]{../images/landscape.jpg}}
%\usebackgroundtemplate{\includegraphics[width=\paperwidth]{../images/tree.jpg}}

\begin{frame}
	\frametitle{Extra: Dockerfile and docker-compose.yml}

	Check out the docker container and framework, see how all that works.

\end{frame}


\begin{frame}
	\frametitle{Extra: Connect it to your GKE cluster}

	I can demo this or we can try it if we have time.

\end{frame}

\begin{frame}
	\frametitle{Extra: GNU Autotools}

	Wow we must be super bored let\'s play with GNU Autotools.

\end{frame}

\begin{frame}
	\frametitle{Future: Scan the PR comments for commands}

	The Cloud Function could monitor the PR for certain strings, using these to trigger actions.

\end{frame}

\begin{frame}
	\frametitle{Resources}
	\href{https://www.codemash.org/session-details/?id=375030}{Click here for Session Details}
	\vspace{2mm}

	Project source files are available: \url{https://github.com/devsecfranklin/workshop-codemash-2023}
	\vspace{2mm}

	Prerequisites are \href{https://prereqs.codemash.org/}{available at this link}.
\end{frame}

\begin{frame}
	\frametitle{Contact}
	\begin{columns}
		\begin{column}{0.5\textwidth}
			Mastodon: ``@devsecfranklin@defcon.social''
			\vspace{2mm}

			E-mail: \textbf{\href{mailto:devsecfranklin@duck.com}{devsecfranklin@duck.com}}
		\end{column}
		\begin{column}{0.5\textwidth}
			\begin{center}
				\includegraphics[width=0.785\textwidth]{../images/rilla.jpg}
			\end{center}
		\end{column}
	\end{columns}
\end{frame}

\end{document}