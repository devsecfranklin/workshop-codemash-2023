% !TeX root = workshop-codemash-2023.tex
% !TeX TXS-program:compile = txs:///pdflatex/[--shell-escape]
% !TeX encoding = UTF-8
% !TeX spellcheck = en_US
% https://orcid.org/0000-0003-4586-8500

% Other possible values are: 1610, 149, 54, 43 and 32.
% By default, it is to 128mm by 96mm(4:3)
\documentclass[aspectratio=169]{beamer} 
%\setbeamertemplate{headline navigation symbols}{}         % no navigation symbols
\usetheme{Warsaw}
\usecolortheme{seahorse}
\usepackage[absolute,overlay]{textpos} % Text positioning


%Information to be included in the title page:
\title{Build a Serverless Github Bot in GCP}
\subtitle{Find and encrypt local secrets}
\author{Franklin Diaz}
\institute{DE:AD:10:C5}
\date{Tuesday January 10, 2023}

\begin{document}

%%% Title Slide %%%
\frame{\titlepage}


%\begin{frame}
%        \frametitle{Table of Contents}
%        \tableofcontents
%\end{frame}

% you can uncomment one of these for the whole doc, or add at the start of each section as desired                                                                            
\usebackgroundtemplate{\includegraphics[width=\paperwidth]{../images/field.jpg}}
%\usebackgroundtemplate{\includegraphics[width=\paperwidth]{../images/landscape.jpg}}
%\usebackgroundtemplate{\includegraphics[width=\paperwidth]{../images/tree.jpg}}

\begin{frame}
    \frametitle{Outline}
    \begin{comment}
    Project source files are available: \url{https://github.com/devsecfranklin/workshop-codemash-2023}
    \end{comment}
    \vspace{2mm}
\end{frame}

\begin{frame}
    \frametitle{Outline}

    A high level overview of the learning path is as follows:

    \begin{raggedright}
        \begin{itemize}
            \item Prerequisites
            \item Github setup.
            \item Set up a development environment.
            \item Review the Python source for the bot.
            \item Configure Terraform and deploy the bot.
            \item Test it out.
            \item Explore possibilities for extending the functionality.
        \end{itemize}
    \end{raggedright}
    \vspace{2mm}
\end{frame}

\begin{frame}
    \frametitle{Setting Up}
    \begin{itemize}
        \item Create a repository on GitHub or other RCS.
        \item Install the tools with script.
    \end{itemize}
\end{frame}

\begin{frame}
    \frametitle{Finding Tokens}
    \begin{itemize}
        \item Scan your local machine for tokens and creds.
    \end{itemize}
\end{frame}

\begin{frame}
    \frametitle{Saving Tokens}
    \begin{itemize}
        \item Encrypt the tokens and push them into RCS.
    \end{itemize}
\end{frame}

\begin{frame}
    \frametitle{Using Tokens}
    \begin{itemize}
        \item Now you can use the secrets in your project without exposing them.
    \end{itemize}
\end{frame}

\begin{frame}
    \frametitle{Considerations}
    \begin{itemize}
        \item You need your GPG key on the local machine to encrypt, decrypt, and use the secrets.
    \end{itemize}
\end{frame}

\end{document}⏎