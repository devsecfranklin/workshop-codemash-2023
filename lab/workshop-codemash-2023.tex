% !TeX document-id = {49bba47f-fadc-40c0-b34f-537b11aff166}
% !TeX root = workshop-codemash-2023.tex
% !TeX TXS-program:compile = txs:///pdflatex/[--shell-escape]

% https://orcid.org/0000-0003-4586-8500

%\setbeamertemplate{headline navigation symbols}{}         % no navigation symbols
\usetheme{Warsaw}
\usecolortheme{seahorse}
\usepackage[absolute,overlay]{textpos} % Text positioning

%%%%%%%%%%%%%%%%%%% TEXT COLOR %%%%%%%%%%%%%%%%%
\usepackage{xcolor}
\definecolor{olive}{rgb}{0.3, 0.4, .1}
\definecolor{fore}{RGB}{249,242,215}
\definecolor{back}{RGB}{51,51,51}
\definecolor{title}{RGB}{255,0,90}
\definecolor{dgreen}{rgb}{0.,0.6,0.}
\definecolor{gold}{rgb}{1.,0.84,0.}
\definecolor{JungleGreen}{cmyk}{0.99,0,0.52,0}
\definecolor{BlueGreen}{cmyk}{0.85,0,0.33,0}
\definecolor{RawSienna}{cmyk}{0,0.72,1,0.45}
\definecolor{Magenta}{cmyk}{0,1,0,0}

%Information to be included in the title page:
\title{Build a Serverless Github Bot in GCP}
\subtitle{}
\author{Franklin Diaz}
\institute{DE:AD:10:C5}
\date{Tuesday January 10, 2023}

%%%%%%%%%%%%%%%%%%% SLIDE VIEW SETTINGS %%%%%%%%%%%%%%%%%
\hypersetup{pdfpagemode=FullScreen} % put the deck in presentation mode
% this is to make notes mode work with pympress package
\usepackage{pgfpages}
\setbeameroption{show notes on second screen}


\begin{document}

\title{\mytitle}
\author[1,2]{Franklin E. Diaz\\ \texttt\href{emailto: frank378@gmail.com}{frank378@gmail.com}}
\affil[1]{Palo Alto Networks}
\begin{titlepage}
	\maketitle
\begin{abstract}
	Did you ever wonder how the cool kids get their bots going to manage pull requests in Github?
	The bots that can comment on Pull Requests, label things, perform other actions that are helpful
	to human developers? Well so did I, so I assembled one a while back that I want to share with you.
\end{abstract}
\end{titlepage}

\begin{comment}
Source files for this document are available at: \url{https://github.com/devsecfranklin/workshop-code-mash-2023/tree/main/}
\end{comment}


\section{\label{sec:Start}Build a Serverless Github Bot in GCP}
\vspace{2mm}

\justifying
This workshop is meant to be a fun way to learn more about some modern software development technologies. You don't need to have a deep understanding of all parts. Rather, you can follow along from end to end and choose to focus more deeply on any part that holds your attention.

\justifying
A high level overview of the learning path is as follows:

\begin{raggedright}
	\begin{enumerate}
		\item Review the Python source for the bot.
		\item Configure Google Cloud account, service user, etc.
		\item Set up GitHub with a bot account, dev token, and webhook.
		\item Configure Terraform and deploy the bot.
		\item Test it out.
		\item Explore possibilities for extending the functionality.
	\end{enumerate}
\end{raggedright}
\vspace{2mm}

\section{\label{sec:preparation}Preparation}

\justifying
There are some requirements that must be met to successfully complete this workshop.

https://cloud.google.com/sdk/docs/install

\subsection{\label{sec:account}Google Cloud Account}

\section{\label{sec:autotools}GNU Autotools}

\justifying
GNU Autotools have been included in the code base for the workshop. Autotools are a well-maintained set of Open Source tools with a gentle learning curve and are included in the distribution of many Open Source packages that we all rely on daily, at least indirectly, and often unknowingly.

\justifying
The list of tools for using this Autotools configuration paradigm is show in Table \ref{Autotools}.
\vspace{2mm}

\begin{table}[ht]
	\centering
	\begin{tabular}{|l|l|}\hline
		Tool & Description \\\hline
		autoconf & Generates a configure script from configure.ac   \\\hline
		automake & Generates a system-specific Makefile based on Makefile.am template    \\\hline
		make  &   X    \\\hline
	\end{tabular}
	\caption{Tools used in this project}
	\label{Autotools}
\end{table}
\vspace{2mm}

\justifying
The image shown in Figure \ref{diagram} is from \cite{autobasics}.
It illustrates the relationship between components mentioned in Table \ref{Autotools}.
\vspace{2mm}

\begin{figure}[ht]
	\includegraphics[width=12cm]{images/diagram.png}
	\caption{A basic overview of how the main Autotools components fit together.}
	\label{diagram}
\end{figure}
\vspace{2mm}

\section{\label{sec:bot}Bot Setup}

Invite bot user ``bot-account'' as a collaborator on the repo.

\section{\label{sec:Python}Python}

\justifying
The Python code is meant to run as a ``Cloud Function'' in GCP. This is a cost-effective way to run code without deploying cloud infrastructure such as dedicated host instances or other cloud infrastructure that must be maintained. We get to run our code and let Google take care of the rest.

\section{\label{sec:Terraform}Terraform}


\clearpage
\begin{versionhistory}
	\vhEntry{v0.1}{October 2nd, 2022}{Franklin Diaz}{Initial Draft}
\end{versionhistory}

\clearpage
% \nocite{*}
\bibliographystyle{plain}
\bibliography{mybib.bib}

\end{document}
